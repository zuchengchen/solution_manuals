\documentclass[10pt,a4paper,twocolumn]{ctexart}
\setlength{\columnseprule}{0.1pt}
%,twocolumn
%\usepackage[latin1]{inputenc}
\usepackage{amsmath}
\usepackage{amsfonts}
\usepackage{amssymb}
\usepackage{graphicx}
\newcommand{\newc}{\newcommand}
\newc{\be}{\begin{equation}}
\newc{\ee}{\end{equation}}
\newc{\Lra}{\Longrightarrow\quad}
\newc{\lvl}{\Bigg|}
\newc{\bc}{\because\ }
\newc{\so}{\therefore\ }
\newc{\om}{\omega}
\newc{\al}{\alpha}
\newc{\Dt}{\Delta}
\newc{\ld}{\lambda}
\newc{\eo}{\varepsilon_0}
\newc{\tht}{\theta}
\newc{\pt}{\partial}
\numberwithin{equation}{subsection} 
\renewcommand{\theequation}{\arabic{subsection}.\arabic{equation}}
%%%%%%%%%%%%%%%%%%%%%%%%%%%%%%%%%%%%%%%%%%%%%%%%%%%%%%%%%%%%%%%%%%%%%%
\usepackage[colorlinks=true,%        color link				   %
citecolor=blue,%		  % cite color							   %
linkcolor=blue,%		  % link color							   %
urlcolor=blue%			  % url color							   %
]{hyperref}				  % create hyperlinks		 			   %
%%%%%%%%%%%%%%%%%%%%%%%%%%%%%%%%%%%%%%%%%%%%%%%%%%%%%%%%%%%%%%%%%%%%%%
%%%%%%%%%%%%%%%%%%%%%%%%%%%%% Page Layout %%%%%%%%%%%%%%%%%%%%%%%%%%%%
\usepackage[%
paper=a4paper,%
top=2cm,% 上 3.5cm %
bottom=2cm,% 下 2.5cm %
left=1.5cm,% 左 2.7cm %
right=1.5cm,% 右 2.7cm %
%headheight=1.0cm,% 页眉 2.5cm %
%footskip=0.7cm% 页脚 1.8cm %
]{geometry} % 页面设置 %
%%%%%%%%%%%%%%%%%%%%%%%%%%%%%%%%%%%%%%%%%%%%%%%%%%%%%%%%%%%%%%%%%%%%%%
\begin{document}
\title{\Large{湖南师范大学数达学院}\\数学物理方法课后习题集解答}

\date{\today}
\author{陈祖成}
\maketitle

%%%%%%%%%%%%%%%%%%%%%%%%%%%%%%%%%%%%%%%%%%%%%%%%%%%%%%%%%%%%%%%%%%%%%%
\begin{abstract}
	此系四川大学数学学院高等数学、微分方程教研室编《高等数学--物理类专用(第四册)(第四版)》所有课后习题的详细答案。
	未完成草稿,如有错误,欢迎纠正。
\end{abstract}
%%%%%%%%%%%%%%%%%%%%%%%%%%%%%%%%%%%%%%%%%%%%%%%%%%%%%%%%%%%%%%%%%%%%%%
\tableofcontents
%%%%%%%%%%%%%%%%%%%%%%%%%%%%%%%%%%%%%%%%%%%%%%%%%%%%%%%%%%%%%%%%%%%%%%
\section{复数与复变函数}	
%%%%%%%%%%%%%%%%%%%%%%%%%%%%%%%%%%%%%%%%%%%%%%%%%%%%%%%%%%%%%%%%%%%%%%

%%%%%%%%%%%%%%%%%%%%%%%%%%%%%%%%%%%%%%%%%%%%%%%%%%%%%%%%%%%%%%%%%%%%%%
\subsection{计算:}
\subsubsection{$(\sqrt{2}- i) - i (1 - i\sqrt{2})$}
解:
	\begin{align}
		& (\sqrt{2}-i)-i(1-i \sqrt{2}) \\
		= & \sqrt{2}-i-i + i^2 \sqrt{2} \\
		= & \sqrt{2}-2i-\sqrt{2} \\
		= & -2 i
	\end{align}

%%%%%%%%%%%%%%%%%%%%%%%%%%%%%%%%%%%%%%%%%%%%%%%%%%%%%%%%%%%%%%%%%%%%%%
\subsubsection{$\frac{1+2 i}{3-4 i}+\frac{2-i}{5 i}$}
解:
\begin{align}
		& \frac{1+2 i}{3-4 i}+\frac{2-i}{5 i} \\
		= & \frac{(1+2 i) 5 i+(2-i)(3-4 i)}{(3-4 i) 5 i} \\
		= & \frac{5 i + 10i^2 +6-8 i-3 i + 4i^2}{5(3 i-4i^2)} \\
		= & \frac{5 i-10+6-11 i-4}{5(3 i+4)} \\
		= & \frac{-6 i-8}{5(3 i+4)} \\
		= & -\frac{2(4+3 i)}{5(4+3 i)} \\
		= & -\frac{2}{5}
	\end{align}
	
%%%%%%%%%%%%%%%%%%%%%%%%%%%%%%%%%%%%%%%%%%%%%%%%%%%%%%%%%%%%%%%%%%%%%%
\subsubsection{$\frac{5}{(1-i)(2-i)(3-i)}$}
\begin{align}
		& \frac{5}{(1-i)(2-i)(3-i)} \\
		= & \frac{5(1+i)(2+i)(3+i)}{(1-i)(2-i)(3-i)(1+i)(2+i)(3+i)} \\
		= & \frac{5(2+i+2 i-1)(3+i)}{(1^2-i^2) \times (2^2-i^2) \times (3^2-i^2)} \\
		= & \frac{5(1+3 i)(3+i)}{2 \times 5 \times 10} \\
		= & \frac{(1+3 i)(3+i)}{20} \\
		= & \frac{3+i+9 i-3}{20} \\
		= & \frac{10 i}{20} \\
		= & \frac{i}{2}
	\end{align}

%%%%%%%%%%%%%%%%%%%%%%%%%%%%%%%%%%%%%%%%%%%%%%%%%%%%%%%%%%%%%%%%%%%%%%
\subsubsection{$(1-i)^4$}
\begin{align}
		& (1-i)^4 \\
		= & {\left[(1-i)^2\right]^2 } \\
		= & {[1-2 i-1]^2 } \\
		= & 4 i^2 \\
		= & -4
	\end{align}


%%%%%%%%%%%%%%%%%%%%%%%%%%%%%%%%%%%%%%%%%%%%%%%%%%%%%%%%%%%%%%%%%%%%%%
\subsection{求下列复数的实部$u$与虚部$v$,模$r$与辐角$\theta$:}

%%%%%%%%%%%%%%%%%%%%%%%%%%%%%%%%%%%%%%%%%%%%%%%%%%%%%%%%%%%%%%%%%%%%%%
\subsubsection{$\frac{1-2 i}{3-4 i}-\frac{2-i}{5 i}$}
\begin{align}
		& \frac{1-2 i}{3-4 i}-\frac{2-i}{5 i} \\
		& =\frac{(1-2 i) 5 i-(2-i)(3-4 i)}{(3-4 i) 5 i} \\
		& =\frac{5 i+10-(6-8 i-3 i-4)}{5(3 i+4)} \\
		& =\frac{5 i+10-2+11 i}{5(4+3 i)} \\
		& =\frac{8+16 i}{5(4+3 i)} \\
		& =\frac{8(1+2 i)(4-3 i)}{5(4+3 i)(4-3 i)} \\
		& =\frac{8(4-3 i+8 i+6)}{5 \times 25} \\
		& =\frac{8(10+5 i)}{5 \times 25} \\
		& =\frac{8(2+i)}{25} \\
		& =\frac{16}{25}+\frac{8}{25} i \\
		& = \frac{8}{25}\sqrt{5} e^{i(\arctan \frac{1}{2} + 2k\pi)}
	\end{align}
所以$u=\frac{16}{25}$, $v=\frac{8}{25}$, $r=\frac{8}{25}\sqrt{5}$, $\theta = \arctan \frac{1}{2} + 2k\pi$。

%%%%%%%%%%%%%%%%%%%%%%%%%%%%%%%%%%%%%%%%%%%%%%%%%%%%%%%%%%%%%%%%%%%%%%
\subsubsection{$(\frac{1+\sqrt{3} i}{2})^n, n=2, 3, 4$}
首先$\frac{1+\sqrt{3} i}{2} = e^{i(\frac{\pi}{3}+ 2m\pi)}$
\begin{itemize}
	\item n=2时
	\begin{align}
		& \left(\frac{1+\sqrt{3} i}{2}\right)^2 \\
		= & (e^{i(\frac{\pi}{3}+ 2m\pi)})^2 \\
		= & e^{i(\frac{2\pi}{3}+ 4m\pi)} \\		
		= & e^{i(\frac{2\pi}{3}+ 4m\pi + 2k\pi)} \\
		= & e^{i(\frac{2\pi}{3}+ 2k\pi)} \\
		= & -\frac{1}{2}+\frac{\sqrt{3}}{2} i
	\end{align}
	所以$u=-\frac{1}{2}$, $v=\frac{\sqrt{3}}{2}$, $r=1$, $\theta = \frac{2\pi}{3} + 2k\pi$。
	
	
	\item n=3时
	\begin{align}
		& \left(\frac{1+\sqrt{3} i}{2}\right)^3 \\
		= & (e^{i(\frac{\pi}{3}+ 2m\pi)})^3 \\
		= & e^{i(\pi+ 6m\pi)} \\
		= & e^{i(\pi+ 2k\pi)} \\
		= & -1
	\end{align}
	所以$u=-1$, $v=0$, $r=1$, $\theta = \pi+ 2k\pi$。
	
	
	\item n=4时
	\begin{align}
		& \left(\frac{1+\sqrt{3} i}{2}\right)^4 \\
		= & (e^{i(\frac{\pi}{3}+ 2m\pi)})^4 \\
		= & e^{i(\frac{4\pi}{3}+ 8m\pi)} \\		
		= & e^{i(-\frac{2\pi}{3} + 2k\pi)} \\
		= & -\frac{1}{2}-\frac{\sqrt{3}}{2} i 
	\end{align}
	所以$u=-\frac{1}{2}$, $v=-\frac{\sqrt{3}}{2}$, $r=1$, $\theta = -\frac{2\pi}{3} + 2k\pi$。
\end{itemize}

%%%%%%%%%%%%%%%%%%%%%%%%%%%%%%%%%%%%%%%%%%%%%%%%%%%%%%%%%%%%%%%%%%%%%%
\subsubsection{$\sqrt{1+i}$}
\begin{align}
	& \sqrt{1+i} \\
	& =(1+i)^{\frac{1}{2}}\\
	& =\left[\sqrt{2} e^{i\left(\frac{\pi}{4}+2 k \pi\right)}\right]^{\frac{1}{2}} \\
	& =2^{\frac{1}{4}} e^{i\left(\frac{\pi}{8}+k \pi\right)} \\
	& =2^{\frac{1}{4}}\left[\cos \left(\frac{\pi}{8}+k \pi\right)+i \sin \left(\frac{\pi}{8}+k \pi\right)\right]
\end{align}
所以$u=2^{\frac{1}{4}} \cos \left(\frac{\pi}{8}+k \pi\right)$, $v=2^{\frac{1}{4}} \sin \left(\frac{\pi}{8}+k \pi\right)$, $r=2^{\frac{1}{4}}$, $\theta = \frac{\pi}{8}+k \pi$。

\end{document}